\chapter*{Заключение}
\addcontentsline{toc}{chapter}{Заключение}

В ходе выполнения данной работы были рассмотрены классификации систем автоматического распознавания речи, возможные дефекты и нарушения речи, изучены существующие речевые характеристики, методы извлечения частотных характеристик.

Среди методов извлечения было уделено внимание коэффициентам линейного предсказания, MFCC-коэффициентам и LPCC-коэффициентам, а также вейвлет-преобразованию. Мел-частотные кепстральные коэффициенты в последнее время приобрели высокую популярность и достаточно эффективно используются в задаче автоматического распознавания речи. Дискретное вейвлет-преобразование же дает возможность расчета большего количества признаков, что позволит выделить из сигнала достаточно информации для дальнейшей работы.

На основе проделанной работы можно сделать вывод о том, что наиболее подходящий метод извлечения признаков для решения  задачи распознавания речи у людей с дефектами речи является дискретное вейвлет-преобразование, так как данный метод позволяет рассчитать большое количество коэффициентов поступившего на вход сигнала для определения более полной информации о нем. 
