\chapter*{Введение}
\setcounter{page}{2}
\addcontentsline{toc}{chapter}{Введение}

Речь -- это самый распространенный вид индивидуального общения. Обработка речи -- это изучения языковых сигналов. Сигналы обрабатываются в цифровой версии, поэтому обработку речи можно рассматривать как уникальный случай цифровой обработки сигналов \cite{vvegenie}.

Автоматическое распознавания речи -- это процесс, в котором из входного речевого сигнала извлекаются необходимые признаки, затем с помощью этих признаков определяются слова/фразы, которые поступили на вход,  структурная схема системы автоматического распознавания речи представлена на рисунке \ref{img:shema}. То есть такая система позволяет компьютеру понимать слова, которые произносит человек \cite{vvegenie2}.

\img{110mm}{shema}{Структурная схема системы автоматического распознавания речи}

В современном мире существует множество технических средств, которые могут воспринимать произносимые речевые сообщения: мобильные телефоны, автомобили, компьютеры и др. Создание приложений, с помощью которых машины могут разговаривать с человеком, особенно правильно реагируя на разговорную речь, давно начало интересовать ученых и инженеров. Однако в настоящее время такая технология оптимизирована не для всех пользователей.

Системы автоматического распознавания речи широко применяются в медицинских исследованиях, например, когда требуется управлять автономными аппаратами. Важной областью применения систем автоматического распознавания речи является помощью людям с инвалидностью, как для людей с нарушениями речи, так и с проблемами опорно-двигательного аппарата \cite{vvegenie3}.

\textbf{Цель данной научно-исследовательской работы} -- это обоснованный выбор метода для распознавания слитной речи у людей с дефектом речи.

Для достижения поставленной цели необходимо решить следующие задачи:
\begin{itemize}
	\item проанализировать классификацию систем автоматического распознавания речи;
	\item рассмотреть возможные дефекты и нарушения речи;
	\item проанализировать методы извлечения признаков (частотной характеристики);
	\item обосновать выбор метода извлечения признаков.
\end{itemize}


